% Options for packages loaded elsewhere
\PassOptionsToPackage{unicode}{hyperref}
\PassOptionsToPackage{hyphens}{url}
\PassOptionsToPackage{dvipsnames,svgnames,x11names}{xcolor}
%
\documentclass[
  letterpaper,
  DIV=11,
  numbers=noendperiod]{scrartcl}

\usepackage{amsmath,amssymb}
\usepackage{iftex}
\ifPDFTeX
  \usepackage[T1]{fontenc}
  \usepackage[utf8]{inputenc}
  \usepackage{textcomp} % provide euro and other symbols
\else % if luatex or xetex
  \usepackage{unicode-math}
  \defaultfontfeatures{Scale=MatchLowercase}
  \defaultfontfeatures[\rmfamily]{Ligatures=TeX,Scale=1}
\fi
\usepackage{lmodern}
\ifPDFTeX\else  
    % xetex/luatex font selection
\fi
% Use upquote if available, for straight quotes in verbatim environments
\IfFileExists{upquote.sty}{\usepackage{upquote}}{}
\IfFileExists{microtype.sty}{% use microtype if available
  \usepackage[]{microtype}
  \UseMicrotypeSet[protrusion]{basicmath} % disable protrusion for tt fonts
}{}
\makeatletter
\@ifundefined{KOMAClassName}{% if non-KOMA class
  \IfFileExists{parskip.sty}{%
    \usepackage{parskip}
  }{% else
    \setlength{\parindent}{0pt}
    \setlength{\parskip}{6pt plus 2pt minus 1pt}}
}{% if KOMA class
  \KOMAoptions{parskip=half}}
\makeatother
\usepackage{xcolor}
\setlength{\emergencystretch}{3em} % prevent overfull lines
\setcounter{secnumdepth}{-\maxdimen} % remove section numbering
% Make \paragraph and \subparagraph free-standing
\ifx\paragraph\undefined\else
  \let\oldparagraph\paragraph
  \renewcommand{\paragraph}[1]{\oldparagraph{#1}\mbox{}}
\fi
\ifx\subparagraph\undefined\else
  \let\oldsubparagraph\subparagraph
  \renewcommand{\subparagraph}[1]{\oldsubparagraph{#1}\mbox{}}
\fi

\usepackage{color}
\usepackage{fancyvrb}
\newcommand{\VerbBar}{|}
\newcommand{\VERB}{\Verb[commandchars=\\\{\}]}
\DefineVerbatimEnvironment{Highlighting}{Verbatim}{commandchars=\\\{\}}
% Add ',fontsize=\small' for more characters per line
\usepackage{framed}
\definecolor{shadecolor}{RGB}{241,243,245}
\newenvironment{Shaded}{\begin{snugshade}}{\end{snugshade}}
\newcommand{\AlertTok}[1]{\textcolor[rgb]{0.68,0.00,0.00}{#1}}
\newcommand{\AnnotationTok}[1]{\textcolor[rgb]{0.37,0.37,0.37}{#1}}
\newcommand{\AttributeTok}[1]{\textcolor[rgb]{0.40,0.45,0.13}{#1}}
\newcommand{\BaseNTok}[1]{\textcolor[rgb]{0.68,0.00,0.00}{#1}}
\newcommand{\BuiltInTok}[1]{\textcolor[rgb]{0.00,0.23,0.31}{#1}}
\newcommand{\CharTok}[1]{\textcolor[rgb]{0.13,0.47,0.30}{#1}}
\newcommand{\CommentTok}[1]{\textcolor[rgb]{0.37,0.37,0.37}{#1}}
\newcommand{\CommentVarTok}[1]{\textcolor[rgb]{0.37,0.37,0.37}{\textit{#1}}}
\newcommand{\ConstantTok}[1]{\textcolor[rgb]{0.56,0.35,0.01}{#1}}
\newcommand{\ControlFlowTok}[1]{\textcolor[rgb]{0.00,0.23,0.31}{#1}}
\newcommand{\DataTypeTok}[1]{\textcolor[rgb]{0.68,0.00,0.00}{#1}}
\newcommand{\DecValTok}[1]{\textcolor[rgb]{0.68,0.00,0.00}{#1}}
\newcommand{\DocumentationTok}[1]{\textcolor[rgb]{0.37,0.37,0.37}{\textit{#1}}}
\newcommand{\ErrorTok}[1]{\textcolor[rgb]{0.68,0.00,0.00}{#1}}
\newcommand{\ExtensionTok}[1]{\textcolor[rgb]{0.00,0.23,0.31}{#1}}
\newcommand{\FloatTok}[1]{\textcolor[rgb]{0.68,0.00,0.00}{#1}}
\newcommand{\FunctionTok}[1]{\textcolor[rgb]{0.28,0.35,0.67}{#1}}
\newcommand{\ImportTok}[1]{\textcolor[rgb]{0.00,0.46,0.62}{#1}}
\newcommand{\InformationTok}[1]{\textcolor[rgb]{0.37,0.37,0.37}{#1}}
\newcommand{\KeywordTok}[1]{\textcolor[rgb]{0.00,0.23,0.31}{#1}}
\newcommand{\NormalTok}[1]{\textcolor[rgb]{0.00,0.23,0.31}{#1}}
\newcommand{\OperatorTok}[1]{\textcolor[rgb]{0.37,0.37,0.37}{#1}}
\newcommand{\OtherTok}[1]{\textcolor[rgb]{0.00,0.23,0.31}{#1}}
\newcommand{\PreprocessorTok}[1]{\textcolor[rgb]{0.68,0.00,0.00}{#1}}
\newcommand{\RegionMarkerTok}[1]{\textcolor[rgb]{0.00,0.23,0.31}{#1}}
\newcommand{\SpecialCharTok}[1]{\textcolor[rgb]{0.37,0.37,0.37}{#1}}
\newcommand{\SpecialStringTok}[1]{\textcolor[rgb]{0.13,0.47,0.30}{#1}}
\newcommand{\StringTok}[1]{\textcolor[rgb]{0.13,0.47,0.30}{#1}}
\newcommand{\VariableTok}[1]{\textcolor[rgb]{0.07,0.07,0.07}{#1}}
\newcommand{\VerbatimStringTok}[1]{\textcolor[rgb]{0.13,0.47,0.30}{#1}}
\newcommand{\WarningTok}[1]{\textcolor[rgb]{0.37,0.37,0.37}{\textit{#1}}}

\providecommand{\tightlist}{%
  \setlength{\itemsep}{0pt}\setlength{\parskip}{0pt}}\usepackage{longtable,booktabs,array}
\usepackage{calc} % for calculating minipage widths
% Correct order of tables after \paragraph or \subparagraph
\usepackage{etoolbox}
\makeatletter
\patchcmd\longtable{\par}{\if@noskipsec\mbox{}\fi\par}{}{}
\makeatother
% Allow footnotes in longtable head/foot
\IfFileExists{footnotehyper.sty}{\usepackage{footnotehyper}}{\usepackage{footnote}}
\makesavenoteenv{longtable}
\usepackage{graphicx}
\makeatletter
\def\maxwidth{\ifdim\Gin@nat@width>\linewidth\linewidth\else\Gin@nat@width\fi}
\def\maxheight{\ifdim\Gin@nat@height>\textheight\textheight\else\Gin@nat@height\fi}
\makeatother
% Scale images if necessary, so that they will not overflow the page
% margins by default, and it is still possible to overwrite the defaults
% using explicit options in \includegraphics[width, height, ...]{}
\setkeys{Gin}{width=\maxwidth,height=\maxheight,keepaspectratio}
% Set default figure placement to htbp
\makeatletter
\def\fps@figure{htbp}
\makeatother

\KOMAoption{captions}{tableheading}
\makeatletter
\@ifpackageloaded{tcolorbox}{}{\usepackage[skins,breakable]{tcolorbox}}
\@ifpackageloaded{fontawesome5}{}{\usepackage{fontawesome5}}
\definecolor{quarto-callout-color}{HTML}{909090}
\definecolor{quarto-callout-note-color}{HTML}{0758E5}
\definecolor{quarto-callout-important-color}{HTML}{CC1914}
\definecolor{quarto-callout-warning-color}{HTML}{EB9113}
\definecolor{quarto-callout-tip-color}{HTML}{00A047}
\definecolor{quarto-callout-caution-color}{HTML}{FC5300}
\definecolor{quarto-callout-color-frame}{HTML}{acacac}
\definecolor{quarto-callout-note-color-frame}{HTML}{4582ec}
\definecolor{quarto-callout-important-color-frame}{HTML}{d9534f}
\definecolor{quarto-callout-warning-color-frame}{HTML}{f0ad4e}
\definecolor{quarto-callout-tip-color-frame}{HTML}{02b875}
\definecolor{quarto-callout-caution-color-frame}{HTML}{fd7e14}
\makeatother
\makeatletter
\makeatother
\makeatletter
\makeatother
\makeatletter
\@ifpackageloaded{caption}{}{\usepackage{caption}}
\AtBeginDocument{%
\ifdefined\contentsname
  \renewcommand*\contentsname{Table of contents}
\else
  \newcommand\contentsname{Table of contents}
\fi
\ifdefined\listfigurename
  \renewcommand*\listfigurename{List of Figures}
\else
  \newcommand\listfigurename{List of Figures}
\fi
\ifdefined\listtablename
  \renewcommand*\listtablename{List of Tables}
\else
  \newcommand\listtablename{List of Tables}
\fi
\ifdefined\figurename
  \renewcommand*\figurename{Figure}
\else
  \newcommand\figurename{Figure}
\fi
\ifdefined\tablename
  \renewcommand*\tablename{Table}
\else
  \newcommand\tablename{Table}
\fi
}
\@ifpackageloaded{float}{}{\usepackage{float}}
\floatstyle{ruled}
\@ifundefined{c@chapter}{\newfloat{codelisting}{h}{lop}}{\newfloat{codelisting}{h}{lop}[chapter]}
\floatname{codelisting}{Listing}
\newcommand*\listoflistings{\listof{codelisting}{List of Listings}}
\makeatother
\makeatletter
\@ifpackageloaded{caption}{}{\usepackage{caption}}
\@ifpackageloaded{subcaption}{}{\usepackage{subcaption}}
\makeatother
\makeatletter
\@ifpackageloaded{tcolorbox}{}{\usepackage[skins,breakable]{tcolorbox}}
\makeatother
\makeatletter
\@ifundefined{shadecolor}{\definecolor{shadecolor}{rgb}{.97, .97, .97}}
\makeatother
\makeatletter
\makeatother
\makeatletter
\makeatother
\ifLuaTeX
  \usepackage{selnolig}  % disable illegal ligatures
\fi
\IfFileExists{bookmark.sty}{\usepackage{bookmark}}{\usepackage{hyperref}}
\IfFileExists{xurl.sty}{\usepackage{xurl}}{} % add URL line breaks if available
\urlstyle{same} % disable monospaced font for URLs
\hypersetup{
  colorlinks=true,
  linkcolor={blue},
  filecolor={Maroon},
  citecolor={Blue},
  urlcolor={Blue},
  pdfcreator={LaTeX via pandoc}}

\author{}
\date{}

\begin{document}
\ifdefined\Shaded\renewenvironment{Shaded}{\begin{tcolorbox}[borderline west={3pt}{0pt}{shadecolor}, boxrule=0pt, interior hidden, sharp corners, breakable, enhanced, frame hidden]}{\end{tcolorbox}}\fi

\hypertarget{control-structures-in-r-conditionals-and-loops}{%
\subsubsection{\texorpdfstring{\textbf{Control Structures in R:
Conditionals and
Loops}}{Control Structures in R: Conditionals and Loops}}\label{control-structures-in-r-conditionals-and-loops}}

Control structures in R allow you to manipulate the flow of your
programs, making them more flexible and dynamic. In this chapter, we'll
focus on conditionals and loops, showing you how to implement
decision-making and repetition in your R scripts.

\hypertarget{conditionals}{%
\subsection{Conditionals}\label{conditionals}}

Conditionals in R let you execute different blocks of code based on
specific conditions. The most common conditional structure is the
``if-else'' statement, but there are other variations as well.

\hypertarget{if-else-statements}{%
\subsubsection{If-Else Statements}\label{if-else-statements}}

An ``if-else'' statement allows you to execute different code blocks
depending on whether a condition is \textbf{\texttt{TRUE}} or
\textbf{\texttt{FALSE}}. The general syntax is as follows:

\begin{Shaded}
\begin{Highlighting}[]
\ControlFlowTok{if}\NormalTok{ (}\ConstantTok{TRUE}\NormalTok{) \{}
  \CommentTok{\# code to execute if condition is TRUE}
\NormalTok{\} }\ControlFlowTok{else}\NormalTok{ \{}
  \CommentTok{\# code to execute if condition is FALSE}
\NormalTok{\}}
\end{Highlighting}
\end{Shaded}

\begin{verbatim}
NULL
\end{verbatim}

Let's consider a simple example to illustrate the concept:

\begin{Shaded}
\begin{Highlighting}[]
\NormalTok{x }\OtherTok{\textless{}{-}} \DecValTok{10}

\ControlFlowTok{if}\NormalTok{ (x }\SpecialCharTok{\textgreater{}} \DecValTok{5}\NormalTok{) \{}
  \FunctionTok{print}\NormalTok{(}\StringTok{"x is greater than 5"}\NormalTok{)}
\NormalTok{\} }\ControlFlowTok{else}\NormalTok{ \{}
  \FunctionTok{print}\NormalTok{(}\StringTok{"x is less than or equal to 5"}\NormalTok{)}
\NormalTok{\}}
\end{Highlighting}
\end{Shaded}

\begin{verbatim}
[1] "x is greater than 5"
\end{verbatim}

In this example, the condition \textbf{\texttt{x\ \textgreater{}\ 5}} is
evaluated. Since it's \textbf{\texttt{TRUE}}, the first block of code is
executed, printing ``x is greater than 5.''

\hypertarget{else-if-statements}{%
\subsubsection{Else IF Statements}\label{else-if-statements}}

You can use ``else if'' statements to check multiple conditions in
sequence. This is useful when there are more than two possible outcomes.

\begin{Shaded}
\begin{Highlighting}[]
\NormalTok{score }\OtherTok{\textless{}{-}} \DecValTok{85}

\ControlFlowTok{if}\NormalTok{ (score }\SpecialCharTok{\textgreater{}=} \DecValTok{90}\NormalTok{) \{}
  \FunctionTok{print}\NormalTok{(}\StringTok{"Grade: A"}\NormalTok{)}
\NormalTok{\} }\ControlFlowTok{else} \ControlFlowTok{if}\NormalTok{ (score }\SpecialCharTok{\textgreater{}=} \DecValTok{80}\NormalTok{) \{}
  \FunctionTok{print}\NormalTok{(}\StringTok{"Grade: B"}\NormalTok{)}
\NormalTok{\} }\ControlFlowTok{else}\NormalTok{ \{}
  \FunctionTok{print}\NormalTok{(}\StringTok{"Grade: C"}\NormalTok{)}
\NormalTok{\}}
\end{Highlighting}
\end{Shaded}

\begin{verbatim}
[1] "Grade: B"
\end{verbatim}

In this example, the ``else if'' statement allows us to assign grades
based on a range of scores. If the first condition isn't met, the next
one is checked, and so on, until a condition is \textbf{\texttt{TRUE}}
or the ``else'' block is executed.

\hypertarget{nested-if-statements}{%
\subsubsection{Nested IF statements}\label{nested-if-statements}}

You can also nest ``if'' statements within other ``if'' statements to
create more complex decision-making logic.

\begin{Shaded}
\begin{Highlighting}[]
\NormalTok{age }\OtherTok{\textless{}{-}} \DecValTok{18}
\NormalTok{has\_id }\OtherTok{\textless{}{-}} \ConstantTok{TRUE}

\ControlFlowTok{if}\NormalTok{ (age }\SpecialCharTok{\textgreater{}=} \DecValTok{18}\NormalTok{) \{}
  \ControlFlowTok{if}\NormalTok{ (has\_id) \{}
    \FunctionTok{print}\NormalTok{(}\StringTok{"Allowed to enter"}\NormalTok{)}
\NormalTok{  \} }\ControlFlowTok{else}\NormalTok{ \{}
    \FunctionTok{print}\NormalTok{(}\StringTok{"ID required"}\NormalTok{)}
\NormalTok{  \}}
\NormalTok{\} }\ControlFlowTok{else}\NormalTok{ \{}
  \FunctionTok{print}\NormalTok{(}\StringTok{"Not allowed to enter"}\NormalTok{)}
\NormalTok{\}}
\end{Highlighting}
\end{Shaded}

\begin{verbatim}
[1] "Allowed to enter"
\end{verbatim}

In this example, the outer ``if'' checks if the person is 18 or older.
If true, the nested ``if'' checks if they have a valid ID. This kind of
nesting is useful when you need to apply additional conditions within a
broader context.

\hypertarget{switch-statements}{%
\subsubsection{Switch Statements}\label{switch-statements}}

A ``switch'' statement allows you to execute different code blocks based
on the value of a variable. This is similar to ``if-else'' but often
more readable when dealing with multiple cases.

\begin{Shaded}
\begin{Highlighting}[]
\NormalTok{day }\OtherTok{\textless{}{-}} \StringTok{"Tuesday"}

\ControlFlowTok{switch}\NormalTok{(day,}
  \StringTok{"Monday"} \OtherTok{=} \FunctionTok{print}\NormalTok{(}\StringTok{"Start of the workweek"}\NormalTok{),}
  \StringTok{"Friday"} \OtherTok{=} \FunctionTok{print}\NormalTok{(}\StringTok{"End of the workweek"}\NormalTok{),}
  \StringTok{"Saturday"} \OtherTok{=} \FunctionTok{print}\NormalTok{(}\StringTok{"Weekend!"}\NormalTok{),}
  \StringTok{"Sunday"} \OtherTok{=} \FunctionTok{print}\NormalTok{(}\StringTok{"Weekend!"}\NormalTok{),}
  \FunctionTok{print}\NormalTok{(}\StringTok{"A regular weekday"}\NormalTok{)}
\NormalTok{)}
\end{Highlighting}
\end{Shaded}

\begin{verbatim}
[1] "A regular weekday"
\end{verbatim}

In this example, the \textbf{\texttt{switch()}} function selects a code
block to execute based on the value of \textbf{\texttt{day}}. If the
value doesn't match any specified cases, the default code block is
executed (the last statement without a named case).

\hypertarget{loops}{%
\subsection{Loops}\label{loops}}

Loops are a fundamental concept in programming, allowing you to repeat
code blocks multiple times. In R, loops come in various forms, including
``for,'' ``while,'' and ``repeat'' loops. Let's explore these in detail.

\hypertarget{for-loops}{%
\subsubsection{For Loops}\label{for-loops}}

A ``for'' loop iterates over a sequence, executing a code block for each
element. This is useful when you know the exact number of iterations.

\begin{Shaded}
\begin{Highlighting}[]
\CommentTok{\# Iterate over a vector}
\NormalTok{vec }\OtherTok{\textless{}{-}} \FunctionTok{c}\NormalTok{(}\DecValTok{1}\NormalTok{, }\DecValTok{2}\NormalTok{, }\DecValTok{3}\NormalTok{, }\DecValTok{4}\NormalTok{, }\DecValTok{5}\NormalTok{)}

\ControlFlowTok{for}\NormalTok{ (i }\ControlFlowTok{in}\NormalTok{ vec) \{}
  \FunctionTok{print}\NormalTok{(}\FunctionTok{paste}\NormalTok{(}\StringTok{"Value:"}\NormalTok{, i))}
\NormalTok{\}}
\end{Highlighting}
\end{Shaded}

\begin{verbatim}
[1] "Value: 1"
[1] "Value: 2"
[1] "Value: 3"
[1] "Value: 4"
[1] "Value: 5"
\end{verbatim}

In this example, the loop iterates over the vector
\textbf{\texttt{vec}}, printing each value with a prefix ``Value:''. You
can also use ``for'' loops with other sequences, such as character
vectors or list indices.

\begin{Shaded}
\begin{Highlighting}[]
\CommentTok{\# Iterate over the names of a data frame}
\NormalTok{df }\OtherTok{\textless{}{-}} \FunctionTok{data.frame}\NormalTok{(}\AttributeTok{Name =} \FunctionTok{c}\NormalTok{(}\StringTok{"Alice"}\NormalTok{, }\StringTok{"Bob"}\NormalTok{, }\StringTok{"Charlie"}\NormalTok{), }\AttributeTok{Age =} \FunctionTok{c}\NormalTok{(}\DecValTok{25}\NormalTok{, }\DecValTok{30}\NormalTok{, }\DecValTok{35}\NormalTok{))}

\ControlFlowTok{for}\NormalTok{ (name }\ControlFlowTok{in} \FunctionTok{names}\NormalTok{(df)) \{}
  \FunctionTok{print}\NormalTok{(}\FunctionTok{paste}\NormalTok{(}\StringTok{"Column:"}\NormalTok{, name))}
\NormalTok{\}}
\end{Highlighting}
\end{Shaded}

\begin{verbatim}
[1] "Column: Name"
[1] "Column: Age"
\end{verbatim}

Here, the loop iterates over the column names of a data frame, printing
the column names one by one.

\hypertarget{while-loops}{%
\subsubsection{While Loops}\label{while-loops}}

A ``while'' loop continues to execute as long as a specified condition
is \textbf{\texttt{TRUE}}. This is useful when the number of iterations
is unknown or dependent on some condition.

\begin{Shaded}
\begin{Highlighting}[]
\CommentTok{\# Example of a simple while loop}
\NormalTok{counter }\OtherTok{\textless{}{-}} \DecValTok{1}

\ControlFlowTok{while}\NormalTok{ (counter }\SpecialCharTok{\textless{}=} \DecValTok{5}\NormalTok{) \{}
  \FunctionTok{print}\NormalTok{(}\FunctionTok{paste}\NormalTok{(}\StringTok{"Counter:"}\NormalTok{, counter))}
\NormalTok{  counter }\OtherTok{\textless{}{-}}\NormalTok{ counter }\SpecialCharTok{+} \DecValTok{1}
\NormalTok{\}}
\end{Highlighting}
\end{Shaded}

\begin{verbatim}
[1] "Counter: 1"
[1] "Counter: 2"
[1] "Counter: 3"
[1] "Counter: 4"
[1] "Counter: 5"
\end{verbatim}

In this example, the loop continues as long as
\textbf{\texttt{counter\ \textless{}=\ 5}}. Once the condition is
\textbf{\texttt{FALSE}}, the loop terminates.

\hypertarget{repeat-loops}{%
\subsubsection{Repeat Loops}\label{repeat-loops}}

A ``repeat'' loop is similar to a ``while'' loop, but it doesn't require
a condition to start. It continues until an explicit
\textbf{\texttt{break}} statement is encountered.

\begin{Shaded}
\begin{Highlighting}[]
\CommentTok{\# Example of a repeat loop with a break condition}
\NormalTok{x }\OtherTok{\textless{}{-}} \DecValTok{0}

\ControlFlowTok{repeat}\NormalTok{ \{}
\NormalTok{  x }\OtherTok{\textless{}{-}}\NormalTok{ x }\SpecialCharTok{+} \DecValTok{1}
  \ControlFlowTok{if}\NormalTok{ (x }\SpecialCharTok{\textgreater{}=} \DecValTok{5}\NormalTok{) \{}
    \ControlFlowTok{break}
\NormalTok{  \}}
  \FunctionTok{print}\NormalTok{(}\FunctionTok{paste}\NormalTok{(}\StringTok{"x is:"}\NormalTok{, x))}
\NormalTok{\}}
\end{Highlighting}
\end{Shaded}

\begin{verbatim}
[1] "x is: 1"
[1] "x is: 2"
[1] "x is: 3"
[1] "x is: 4"
\end{verbatim}

In this example, the loop continues indefinitely until the condition
\textbf{\texttt{x\ \textgreater{}=\ 5}} is met, at which point the
\textbf{\texttt{break}} statement stops the loop.

\hypertarget{nested-loops}{%
\subsubsection{Nested Loops}\label{nested-loops}}

You can nest loops within other loops to handle more complex tasks. This
is useful for working with multi-dimensional data, such as matrices or
data frames.

\begin{Shaded}
\begin{Highlighting}[]
\CommentTok{\# Example of nested loops to iterate over a matrix}
\NormalTok{matrix }\OtherTok{\textless{}{-}} \FunctionTok{matrix}\NormalTok{(}\DecValTok{1}\SpecialCharTok{:}\DecValTok{9}\NormalTok{, }\AttributeTok{nrow =} \DecValTok{3}\NormalTok{, }\AttributeTok{ncol =} \DecValTok{3}\NormalTok{)}

\ControlFlowTok{for}\NormalTok{ (i }\ControlFlowTok{in} \DecValTok{1}\SpecialCharTok{:}\FunctionTok{nrow}\NormalTok{(matrix)) \{}
  \ControlFlowTok{for}\NormalTok{ (j }\ControlFlowTok{in} \DecValTok{1}\SpecialCharTok{:}\FunctionTok{ncol}\NormalTok{(matrix)) \{}
    \FunctionTok{print}\NormalTok{(}\FunctionTok{paste}\NormalTok{(}\StringTok{"Element at ("}\NormalTok{, i, }\StringTok{","}\NormalTok{, j, }\StringTok{") is"}\NormalTok{, matrix[i, j]))}
\NormalTok{  \}}
\NormalTok{\}}
\end{Highlighting}
\end{Shaded}

\begin{verbatim}
[1] "Element at ( 1 , 1 ) is 1"
[1] "Element at ( 1 , 2 ) is 4"
[1] "Element at ( 1 , 3 ) is 7"
[1] "Element at ( 2 , 1 ) is 2"
[1] "Element at ( 2 , 2 ) is 5"
[1] "Element at ( 2 , 3 ) is 8"
[1] "Element at ( 3 , 1 ) is 3"
[1] "Element at ( 3 , 2 ) is 6"
[1] "Element at ( 3 , 3 ) is 9"
\end{verbatim}

In this example, nested loops are used to iterate over the rows and
columns of a matrix, printing each element with its coordinates.

\hypertarget{loop-control-statements}{%
\subsection{Loop Control Statements}\label{loop-control-statements}}

In addition to \textbf{\texttt{break}}, R provides
\textbf{\texttt{next}} to skip an iteration and continue with the next
one, and \textbf{\texttt{return}} to exit a loop and return a value (in
functions).

\begin{Shaded}
\begin{Highlighting}[]
\CommentTok{\# Example of using \textquotesingle{}next\textquotesingle{} to skip even numbers}
\ControlFlowTok{for}\NormalTok{ (i }\ControlFlowTok{in} \DecValTok{1}\SpecialCharTok{:}\DecValTok{10}\NormalTok{) \{}
  \ControlFlowTok{if}\NormalTok{ (i }\SpecialCharTok{\%\%} \DecValTok{2} \SpecialCharTok{==} \DecValTok{0}\NormalTok{) \{  }\CommentTok{\# Check if even}
    \ControlFlowTok{next}  \CommentTok{\# Skip even numbers}
\NormalTok{  \}}
  \FunctionTok{print}\NormalTok{(i)  }\CommentTok{\# Print odd numbers}
\NormalTok{\}}
\end{Highlighting}
\end{Shaded}

\begin{verbatim}
[1] 1
[1] 3
[1] 5
[1] 7
[1] 9
\end{verbatim}

Here, the \textbf{\texttt{next}} statement is used to skip even numbers,
resulting in only odd numbers being printed.

\begin{tcolorbox}[enhanced jigsaw, colback=white, colframe=quarto-callout-tip-color-frame, title=\textcolor{quarto-callout-tip-color}{\faLightbulb}\hspace{0.5em}{Tip}, coltitle=black, toprule=.15mm, breakable, toptitle=1mm, bottomtitle=1mm, opacitybacktitle=0.6, leftrule=.75mm, titlerule=0mm, arc=.35mm, rightrule=.15mm, colbacktitle=quarto-callout-tip-color!10!white, bottomrule=.15mm, left=2mm, opacityback=0]

\end{tcolorbox}



\end{document}
